% Options for packages loaded elsewhere
\PassOptionsToPackage{unicode}{hyperref}
\PassOptionsToPackage{hyphens}{url}
%
\documentclass[
]{article}
\usepackage{amsmath,amssymb}
\usepackage{iftex}
\ifPDFTeX
  \usepackage[T1]{fontenc}
  \usepackage[utf8]{inputenc}
  \usepackage{textcomp} % provide euro and other symbols
\else % if luatex or xetex
  \usepackage{unicode-math} % this also loads fontspec
  \defaultfontfeatures{Scale=MatchLowercase}
  \defaultfontfeatures[\rmfamily]{Ligatures=TeX,Scale=1}
\fi
\usepackage{lmodern}
\ifPDFTeX\else
  % xetex/luatex font selection
\fi
% Use upquote if available, for straight quotes in verbatim environments
\IfFileExists{upquote.sty}{\usepackage{upquote}}{}
\IfFileExists{microtype.sty}{% use microtype if available
  \usepackage[]{microtype}
  \UseMicrotypeSet[protrusion]{basicmath} % disable protrusion for tt fonts
}{}
\makeatletter
\@ifundefined{KOMAClassName}{% if non-KOMA class
  \IfFileExists{parskip.sty}{%
    \usepackage{parskip}
  }{% else
    \setlength{\parindent}{0pt}
    \setlength{\parskip}{6pt plus 2pt minus 1pt}}
}{% if KOMA class
  \KOMAoptions{parskip=half}}
\makeatother
\usepackage{xcolor}
\usepackage[margin=1in]{geometry}
\usepackage{color}
\usepackage{fancyvrb}
\newcommand{\VerbBar}{|}
\newcommand{\VERB}{\Verb[commandchars=\\\{\}]}
\DefineVerbatimEnvironment{Highlighting}{Verbatim}{commandchars=\\\{\}}
% Add ',fontsize=\small' for more characters per line
\usepackage{framed}
\definecolor{shadecolor}{RGB}{248,248,248}
\newenvironment{Shaded}{\begin{snugshade}}{\end{snugshade}}
\newcommand{\AlertTok}[1]{\textcolor[rgb]{0.94,0.16,0.16}{#1}}
\newcommand{\AnnotationTok}[1]{\textcolor[rgb]{0.56,0.35,0.01}{\textbf{\textit{#1}}}}
\newcommand{\AttributeTok}[1]{\textcolor[rgb]{0.13,0.29,0.53}{#1}}
\newcommand{\BaseNTok}[1]{\textcolor[rgb]{0.00,0.00,0.81}{#1}}
\newcommand{\BuiltInTok}[1]{#1}
\newcommand{\CharTok}[1]{\textcolor[rgb]{0.31,0.60,0.02}{#1}}
\newcommand{\CommentTok}[1]{\textcolor[rgb]{0.56,0.35,0.01}{\textit{#1}}}
\newcommand{\CommentVarTok}[1]{\textcolor[rgb]{0.56,0.35,0.01}{\textbf{\textit{#1}}}}
\newcommand{\ConstantTok}[1]{\textcolor[rgb]{0.56,0.35,0.01}{#1}}
\newcommand{\ControlFlowTok}[1]{\textcolor[rgb]{0.13,0.29,0.53}{\textbf{#1}}}
\newcommand{\DataTypeTok}[1]{\textcolor[rgb]{0.13,0.29,0.53}{#1}}
\newcommand{\DecValTok}[1]{\textcolor[rgb]{0.00,0.00,0.81}{#1}}
\newcommand{\DocumentationTok}[1]{\textcolor[rgb]{0.56,0.35,0.01}{\textbf{\textit{#1}}}}
\newcommand{\ErrorTok}[1]{\textcolor[rgb]{0.64,0.00,0.00}{\textbf{#1}}}
\newcommand{\ExtensionTok}[1]{#1}
\newcommand{\FloatTok}[1]{\textcolor[rgb]{0.00,0.00,0.81}{#1}}
\newcommand{\FunctionTok}[1]{\textcolor[rgb]{0.13,0.29,0.53}{\textbf{#1}}}
\newcommand{\ImportTok}[1]{#1}
\newcommand{\InformationTok}[1]{\textcolor[rgb]{0.56,0.35,0.01}{\textbf{\textit{#1}}}}
\newcommand{\KeywordTok}[1]{\textcolor[rgb]{0.13,0.29,0.53}{\textbf{#1}}}
\newcommand{\NormalTok}[1]{#1}
\newcommand{\OperatorTok}[1]{\textcolor[rgb]{0.81,0.36,0.00}{\textbf{#1}}}
\newcommand{\OtherTok}[1]{\textcolor[rgb]{0.56,0.35,0.01}{#1}}
\newcommand{\PreprocessorTok}[1]{\textcolor[rgb]{0.56,0.35,0.01}{\textit{#1}}}
\newcommand{\RegionMarkerTok}[1]{#1}
\newcommand{\SpecialCharTok}[1]{\textcolor[rgb]{0.81,0.36,0.00}{\textbf{#1}}}
\newcommand{\SpecialStringTok}[1]{\textcolor[rgb]{0.31,0.60,0.02}{#1}}
\newcommand{\StringTok}[1]{\textcolor[rgb]{0.31,0.60,0.02}{#1}}
\newcommand{\VariableTok}[1]{\textcolor[rgb]{0.00,0.00,0.00}{#1}}
\newcommand{\VerbatimStringTok}[1]{\textcolor[rgb]{0.31,0.60,0.02}{#1}}
\newcommand{\WarningTok}[1]{\textcolor[rgb]{0.56,0.35,0.01}{\textbf{\textit{#1}}}}
\usepackage{graphicx}
\makeatletter
\def\maxwidth{\ifdim\Gin@nat@width>\linewidth\linewidth\else\Gin@nat@width\fi}
\def\maxheight{\ifdim\Gin@nat@height>\textheight\textheight\else\Gin@nat@height\fi}
\makeatother
% Scale images if necessary, so that they will not overflow the page
% margins by default, and it is still possible to overwrite the defaults
% using explicit options in \includegraphics[width, height, ...]{}
\setkeys{Gin}{width=\maxwidth,height=\maxheight,keepaspectratio}
% Set default figure placement to htbp
\makeatletter
\def\fps@figure{htbp}
\makeatother
\setlength{\emergencystretch}{3em} % prevent overfull lines
\providecommand{\tightlist}{%
  \setlength{\itemsep}{0pt}\setlength{\parskip}{0pt}}
\setcounter{secnumdepth}{-\maxdimen} % remove section numbering
\usepackage{fontspec}
\usepackage{amsmath}
\usepackage[utf8]{inputenc}
\ifLuaTeX
  \usepackage{selnolig}  % disable illegal ligatures
\fi
\IfFileExists{bookmark.sty}{\usepackage{bookmark}}{\usepackage{hyperref}}
\IfFileExists{xurl.sty}{\usepackage{xurl}}{} % add URL line breaks if available
\urlstyle{same}
\hypersetup{
  pdftitle={Graded Assignment: Data Analysis Project},
  pdfauthor={Tiago B. Lacerda},
  hidelinks,
  pdfcreator={LaTeX via pandoc}}

\title{Graded Assignment: Data Analysis Project}
\usepackage{etoolbox}
\makeatletter
\providecommand{\subtitle}[1]{% add subtitle to \maketitle
  \apptocmd{\@title}{\par {\large #1 \par}}{}{}
}
\makeatother
\subtitle{Bayesian Statistics Specialization: Course 2, Techniques and
Models}
\author{Tiago B. Lacerda}
\date{2025-05-11}

\begin{document}
\maketitle

\hypertarget{executive-summary}{%
\section{Executive Summary}\label{executive-summary}}

\hypertarget{introduction}{%
\section{Introduction}\label{introduction}}

In modern mobile networks (4G and 5G), operators face the challenge of
allocating limited resources efficiently to maintain and enhance service
quality. A critical performance metric in this context is the Excellent
Consistent Quality (ECQ) test success rate at individual sites. ECQ
assessments evaluate whether networks consistently support demanding
applications such as video streaming, video calls, and gaming, ensuring
a seamless user experience .

These tests are typically conducted on Android or iOS devices with
embedded SDKs in applications, contingent upon user consent. They
measure key performance indicators (KPIs) including download speed,
upload speed, latency, jitter, packet loss, and time to first byte,
aligning with thresholds recommended for various demanding applications
.

However, the variability in the number of tests across sites---some
reporting only a handful while others report hundreds due to natural
user mobility---poses a significant challenge. Naïve ``site-by-site''
estimates can be misleading: small samples may produce extreme rates
simply due to chance, and citywide averages can obscure localized
underperformance.

To address this, we propose a three-level Bayesian hierarchical model
that nests individual sites within municipalities and municipalities
within ANFs. This framework facilitates the sharing of information
across levels, yielding stable, data-driven estimates of each site's
true ECQ success probability while properly quantifying uncertainty.

In this report, we begin by clearly defining the problem and the key
question we aim to answer: Which sites have a true ECQ success
probability significantly below the network average, and how can we rank
them for targeted interventions, accounting for both data scarcity and
local variability? We then detail the model structure, inference
methodology, and decision-making strategy.

\hypertarget{problem-definition}{%
\subsubsection{Problem Definition}\label{problem-definition}}

Our network comprises multiple sites scattered across a city, each
running a varying number of ECQ tests. Some sites may report as few as
5--20 tests in a given period, while others conduct several hundred. The
core challenge is to identify which sites genuinely underperform in
terms of ECQ success rate and thus prioritize network improvement
investments, without being misled by the randomness inherent in small
test counts.

\hypertarget{specific-question}{%
\paragraph{Specific Question}\label{specific-question}}

\begin{quote}
\emph{Which sites have a true ECQ success probability significantly
below the network average, and how can rank them for targeted
interventions, accounting for both data scarcity and local variability?}
\end{quote}

By formalizing this question, we set the stage for applying a
hierarchical Bayes model that ``borrows strength'' across sites and
municipalities, producing posterior distributions for each site's
success probability. These posteriors underpin credible intervals and
ranking metrics that guide robust, data-informed investment decisions.

\hypertarget{data}{%
\subsubsection{Data}\label{data}}

In this report, we analyze ECQ test data collected in October 2024
across Brazil's Northeast region, encompassing 8 ANFs. Each data point
corresponds to a specific network site, identified by its unique
ENDERECO\_ID. For every site, we have recorded the total number of ECQ
tests conducted and the number of successful tests (TESTES\_ECQ\_OK),
indicating instances where the network met the stringent performance
thresholds defined by the ECQ metric.

Below is a summary of the data we will be using in our analysis.

\begin{verbatim}
## tibble [958 x 6] (S3: tbl_df/tbl/data.frame)
##  $ group_id     : int [1:958] 101 103 93 106 104 107 95 96 94 97 ...
##  $ ANF          : chr [1:958] "83" "83" "83" "83" ...
##  $ MUNICIPIO    : chr [1:958] "ARACAGI" "ARARUNA" "AGUA BRANCA" "AREIAL" ...
##  $ ENDERECO_ID  : chr [1:958] "PBAAG_0001" "PBAAN_0001" "PBABW_0001" "PBAEA_0001" ...
##  $ TESTES_ECQ_OK: num [1:958] 9 27 12 10 80 0 19 5 26 0 ...
##  $ TESTES_ECQ   : num [1:958] 13 57 28 12 150 1 21 12 31 12 ...
\end{verbatim}

\includegraphics{report_files/figure-latex/PLOT-1.pdf}

\hypertarget{bayesian-model}{%
\subsubsection{Bayesian Model}\label{bayesian-model}}

\hypertarget{hierarchical-model-specification}{%
\subsubsection{Hierarchical Model
Specification}\label{hierarchical-model-specification}}

\hypertarget{bayesian-model-1}{%
\subsection{Bayesian Model}\label{bayesian-model-1}}

\hypertarget{level-1-site-level-observations}{%
\subsubsection{Level 1: Site-Level
Observations}\label{level-1-site-level-observations}}

For each site \(s\), we observe: \[
y_s \sim \text{Binomial}(n_{\text{tests}, \theta_s)
\] where \(\theta_s\) is the true success probability for site \(s\)

\hypertarget{level-2-municipality-level-parameters}{%
\subsubsection{Level 2: Municipality-Level
Parameters}\label{level-2-municipality-level-parameters}}

\[
\text{logit}(\theta_s) = \text{logit}(\mu_g) + \epsilon_s \\
\epsilon_s \sim \mathcal{N}(0, \phi_{\text{site}}^{-1})
\] where: - \(\mu_g\) = average success probability for municipality
\(g\) - \(\phi_{\text{site}}\) = precision at site level

\hypertarget{level-3-anf-level-parameters}{%
\subsubsection{Level 3: ANF-Level
Parameters}\label{level-3-anf-level-parameters}}

\[
\mu_g \sim \text{Beta}(\alpha_g, \beta_g)
\] where: \[
\begin{aligned}
\alpha_g &= \mu_{\text{ANF}} \cdot \phi_{\text{municipio}} + 0.1 \\
\beta_g &= (1 - \mu_{\text{ANF}}) \cdot \phi_{\text{municipio}} + 0.1
\end{aligned}
\] \emph{(0.1 added for numerical stability)}

\hypertarget{hyperpriors}{%
\subsubsection{Hyperpriors}\label{hyperpriors}}

\begin{Shaded}
\begin{Highlighting}[]
\NormalTok{jags\_data }\OtherTok{\textless{}{-}} \FunctionTok{list}\NormalTok{(}
  \AttributeTok{N\_group =} \FunctionTok{max}\NormalTok{(data}\SpecialCharTok{$}\NormalTok{group\_id),}
  \AttributeTok{N\_sites =} \FunctionTok{nrow}\NormalTok{(data),}
  \AttributeTok{group\_per\_site =}\NormalTok{ data}\SpecialCharTok{$}\NormalTok{group\_id,  }\CommentTok{\# Vetor de grupos por site}
  \AttributeTok{n\_tests =}\NormalTok{ data}\SpecialCharTok{$}\NormalTok{TESTES\_ECQ,       }\CommentTok{\# Vetor de testes}
  \AttributeTok{n\_success =}\NormalTok{ data}\SpecialCharTok{$}\NormalTok{TESTES\_ECQ\_OK   }\CommentTok{\# Vetor de sucessos}
\NormalTok{)}

\NormalTok{model\_string }\OtherTok{\textless{}{-}} \StringTok{"}
\StringTok{model \{}
\StringTok{  \# Hiperparâmetros globais}
\StringTok{  mu\_global \textasciitilde{} dbeta(3, 3)}
\StringTok{  sigma\_global \textasciitilde{} dgamma(2, 0.5)  \# Prior Gamma mais informativa}

\StringTok{  \# Parâmetros ANF com restrição}
\StringTok{  alpha\_anf \textless{}{-} mu\_global * sigma\_global}
\StringTok{  beta\_anf \textless{}{-} (1 {-} mu\_global) * sigma\_global }
\StringTok{  mu\_anf \textasciitilde{} dbeta(alpha\_anf, beta\_anf)}

\StringTok{  \# Priors para dispersão}
\StringTok{  phi\_municipio \textasciitilde{} dgamma(2, 0.9)  \# Gamma mais suave}
\StringTok{  phi\_site \textasciitilde{} dgamma(2, 2)}

\StringTok{  \# Loop por municípios}
\StringTok{  for(g in 1:N\_group) \{}
\StringTok{    a\_municipio[g] \textless{}{-} mu\_anf * phi\_municipio}
\StringTok{    b\_municipio[g] \textless{}{-} (1 {-} mu\_anf) * phi\_municipio}
\StringTok{    mu\_municipio[g] \textasciitilde{} dbeta(a\_municipio[g], b\_municipio[g])}
\StringTok{  \}}

\StringTok{  \# Loop por sites}
\StringTok{  for(s in 1:N\_sites) \{}
\StringTok{    logit\_mu\_site[s] \textless{}{-} logit(mu\_municipio[group\_per\_site[s]])}
\StringTok{    theta\_site[s] \textless{}{-} ilogit(logit\_mu\_site[s] + epsilon[s])}
\StringTok{    epsilon[s] \textasciitilde{} dnorm(0, 1/phi\_site)}
\StringTok{    }
\StringTok{    n\_success[s] \textasciitilde{} dbin(theta\_site[s], n\_tests[s])}
\StringTok{  \}}
\StringTok{\}}
\StringTok{"}


\NormalTok{inits }\OtherTok{\textless{}{-}} \ControlFlowTok{function}\NormalTok{() \{}
  \FunctionTok{list}\NormalTok{(}
    \AttributeTok{mu\_global =} \FunctionTok{rbeta}\NormalTok{(}\DecValTok{1}\NormalTok{, }\DecValTok{3}\NormalTok{, }\DecValTok{3}\NormalTok{),}
    \AttributeTok{sigma\_global =} \FunctionTok{rgamma}\NormalTok{(}\DecValTok{1}\NormalTok{, }\DecValTok{2}\NormalTok{, }\FloatTok{0.5}\NormalTok{),  }\CommentTok{\# Coerente com o prior}
    \AttributeTok{phi\_municipio =} \FunctionTok{rgamma}\NormalTok{(}\DecValTok{1}\NormalTok{, }\DecValTok{2}\NormalTok{, }\DecValTok{2}\NormalTok{),}
    \AttributeTok{phi\_site =} \FunctionTok{rgamma}\NormalTok{(}\DecValTok{1}\NormalTok{, }\DecValTok{2}\NormalTok{, }\DecValTok{2}\NormalTok{),}
    \AttributeTok{mu\_anf =} \FunctionTok{rbeta}\NormalTok{(}\DecValTok{1}\NormalTok{, }\DecValTok{2}\NormalTok{, }\DecValTok{2}\NormalTok{)  }\CommentTok{\# Inicialização direta}
    
\NormalTok{  )}
\NormalTok{\}}

\NormalTok{model }\OtherTok{\textless{}{-}} \FunctionTok{jags.model}\NormalTok{(}
  \FunctionTok{textConnection}\NormalTok{(model\_string),}
  \AttributeTok{data =}\NormalTok{ jags\_data,}
  \AttributeTok{inits =}\NormalTok{ inits,}
  \AttributeTok{n.chains =} \DecValTok{4}\NormalTok{,}
  \AttributeTok{n.adapt =} \DecValTok{5000}  \CommentTok{\# Aumentar fase de adaptação}
\NormalTok{)}
\end{Highlighting}
\end{Shaded}

\begin{verbatim}
## Compiling model graph
##    Resolving undeclared variables
##    Allocating nodes
## Graph information:
##    Observed stochastic nodes: 958
##    Unobserved stochastic nodes: 1258
##    Total graph size: 6358
## 
## Initializing model
\end{verbatim}

\begin{Shaded}
\begin{Highlighting}[]
\NormalTok{samples }\OtherTok{\textless{}{-}} \FunctionTok{coda.samples}\NormalTok{(}
\NormalTok{  model,}
  \AttributeTok{variable.names =} \FunctionTok{c}\NormalTok{(}\StringTok{"mu\_anf"}\NormalTok{, }\StringTok{"mu\_municipio"}\NormalTok{, }\StringTok{"theta\_site"}\NormalTok{),}
  \AttributeTok{n.iter =} \DecValTok{10000}\NormalTok{,}
  \AttributeTok{thin =} \DecValTok{5}
\NormalTok{)}
\end{Highlighting}
\end{Shaded}

\hypertarget{conclusions-summarize-your-conclusions-based-on-the-results-process-step-8.-this-section-may-be-combined-with-the-results-section.}{%
\subsubsection{Conclusions: Summarize your conclusions based on the
results (process step 8). This section may be combined with the results
section.}\label{conclusions-summarize-your-conclusions-based-on-the-results-process-step-8.-this-section-may-be-combined-with-the-results-section.}}

\begin{enumerate}
\def\labelenumi{\arabic{enumi}.}
\setcounter{enumi}{7}
\tightlist
\item
  Use the model
\end{enumerate}

\begin{itemize}
\tightlist
\item
  Provide relevant posterior summaries.
\item
  Interpret the model results in the context of the problem.
\item
  Use the results to reach a conclusion.
\item
  Acknowledge shortcomings of the model or caveats for your results.
\item
  Write a report that does not exceed four pages (including figures and
  tables). It will be challenging to address all of the items above in
  so short a space. It is up to you to decide which parts should be
  emphasized and discussed in detail and which parts should merely be
  summarized in a sentence or two. Remember that a peer reviewer, after
  completing this course, should be able to re-create your results if
  you were to provide the data with your report (which you will not).
\end{itemize}

\end{document}
